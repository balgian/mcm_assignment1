\section{Exercise 1} \label{P1}
% Write some intro
\subsection{Q1.1}
The \textit{AngleAxisToRot} is a function that, compute the \textit{rotation matrix} $\mathbf{R}$ with Rodrigues' formula 
\begin{equation}
	\mathbf{R} = \mathbf{I}_{3\times 3} + \sin{\theta}[\mathbf{h}\times] + (1 - \cos \theta)[\mathbf{h}\times]^2
	\label{eq::rodriguesformula}
\end{equation}
Given the \textit{axis of rotation} $\mathbf{h}$, which checked if it is a unit vector, can be considered three different cases for Equation \ref{eq::rodriguesformula} due to the values of \textit{angle of rotation} $\theta$:
\begin{itemize}
	\item $\theta = 0 \Rightarrow \mathbf{R} = \mathbf{I}_{3\times 3}$\space;
	\item $\theta = \pi \Rightarrow \mathbf{R} = \mathbf{I}_{3\times 3} + 2 [\mathbf{h}\times]^2$ where the square of a skew-symmetric matrix is $[\mathbf{h}\times]^2 = (\mathbf{h}\mathbf{h}^T -\mathbf{I}_{3\times 3})$\space;
	\item $\theta \in (0, \pi)$ where, since there are not simplifications of Equation \ref{eq::rodriguesformula} was needed to compute the axis of rotation's skew-symmetric matrix
		\begin{equation*}
			[\mathbf{h}\times] = \begin{bmatrix}
				0 & -h_3 & h_2 \\
				h_3 & 0 & -h_1 \\
				-h_2 & h_1 & 0
			\end{bmatrix}
	\end{equation*}
\end{itemize}

\subsection{Q1.2}
In the first case $\mathbf{h} = [1, 0, 0]^T$ and $\theta = \pi/2$ represent a $90$-degree rotation around the $x$-axis. In fact, $\mathbf{R}= \mathbf{R}_z\mathbf{R}_y\mathbf{R}_x =\mathbf{I}_{3\times 3}\mathbf{I}_{3\times 3}\mathbf{R}_x = \mathbf{R}_x$ where
\begin{equation*}
	\mathbf{R}_x = \begin{bmatrix}
				1 & 0 & 0 \\
				0 & 0 & -1 \\
				0 & 1 & 0
			\end{bmatrix}
\end{equation*}
\subsection{Q1.3}
In this configuration, which $\mathbf{h} = [0, 0, 1]^T$ and $\theta = \pi/3$, likely the previous case, it is a rotation around the $z$-axis but of $60$-degree
\begin{equation*}
	\mathbf{R}_z = \begin{bmatrix}
				\frac{1}{2} & -\frac{\sqrt{3}}{2} & 0 \\
				\frac{\sqrt{3}}{2} & \frac{1}{2} & 0 \\
				0 & 0 & 1
			\end{bmatrix}
\end{equation*}

\subsection{Q1.4}
For the last case, $\mathbf{\rho} = [-\pi/3, -\pi/6 ,\pi/3]$, before to call the function \textit{AngleAxisToRot} have been computed $\mathbf{h}$ and $\theta$ knowing that $\mathbf{\rho} = \theta\mathbf{h}$, and by definition $\parallel \mathbf{h} \parallel = 1$ this entail $\parallel \mathbf{\rho} \parallel = \parallel \theta \mathbf{h} \parallel = \theta \parallel \mathbf{h} \parallel = \theta$.
\begin{equation*}
	\theta = \sqrt{\rho _x^2 +\rho _y^2 +\rho _z^2} = \frac{\pi}{2} \qquad \mathbf{h} = \frac{\rho}{\theta} = [-\frac{2}{3},-\frac{1}{3},\frac{2}{3}]^T
\end{equation*}
This means that $R$ is a $90$-degree rotation around the vector $h$.
\begin{equation*}
	\mathbf{R} = \begin{bmatrix} 
		\frac{4}{9} & -\frac{4}{9} & -\frac{7}{9} \\
		\frac{8}{9} & \frac{1}{9} & \frac{4}{9} \\
		-\frac{1}{9} & -\frac{8}{9} & \frac{4}{9} 
	\end{bmatrix}
\end{equation*}
